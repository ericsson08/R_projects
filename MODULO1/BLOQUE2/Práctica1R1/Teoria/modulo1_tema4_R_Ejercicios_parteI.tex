\documentclass[]{article}
\usepackage{lmodern}
\usepackage{amssymb,amsmath}
\usepackage{ifxetex,ifluatex}
\usepackage{fixltx2e} % provides \textsubscript
\ifnum 0\ifxetex 1\fi\ifluatex 1\fi=0 % if pdftex
  \usepackage[T1]{fontenc}
  \usepackage[utf8]{inputenc}
\else % if luatex or xelatex
  \ifxetex
    \usepackage{mathspec}
  \else
    \usepackage{fontspec}
  \fi
  \defaultfontfeatures{Ligatures=TeX,Scale=MatchLowercase}
\fi
% use upquote if available, for straight quotes in verbatim environments
\IfFileExists{upquote.sty}{\usepackage{upquote}}{}
% use microtype if available
\IfFileExists{microtype.sty}{%
\usepackage{microtype}
\UseMicrotypeSet[protrusion]{basicmath} % disable protrusion for tt fonts
}{}
\usepackage[margin=1in]{geometry}
\usepackage{hyperref}
\hypersetup{unicode=true,
            pdftitle={MÓDULO 1: HERRAMIENTAS BIG DATA},
            pdfauthor={Nombre Alumno},
            pdfborder={0 0 0},
            breaklinks=true}
\urlstyle{same}  % don't use monospace font for urls
\usepackage{color}
\usepackage{fancyvrb}
\newcommand{\VerbBar}{|}
\newcommand{\VERB}{\Verb[commandchars=\\\{\}]}
\DefineVerbatimEnvironment{Highlighting}{Verbatim}{commandchars=\\\{\}}
% Add ',fontsize=\small' for more characters per line
\usepackage{framed}
\definecolor{shadecolor}{RGB}{248,248,248}
\newenvironment{Shaded}{\begin{snugshade}}{\end{snugshade}}
\newcommand{\AlertTok}[1]{\textcolor[rgb]{0.94,0.16,0.16}{#1}}
\newcommand{\AnnotationTok}[1]{\textcolor[rgb]{0.56,0.35,0.01}{\textbf{\textit{#1}}}}
\newcommand{\AttributeTok}[1]{\textcolor[rgb]{0.77,0.63,0.00}{#1}}
\newcommand{\BaseNTok}[1]{\textcolor[rgb]{0.00,0.00,0.81}{#1}}
\newcommand{\BuiltInTok}[1]{#1}
\newcommand{\CharTok}[1]{\textcolor[rgb]{0.31,0.60,0.02}{#1}}
\newcommand{\CommentTok}[1]{\textcolor[rgb]{0.56,0.35,0.01}{\textit{#1}}}
\newcommand{\CommentVarTok}[1]{\textcolor[rgb]{0.56,0.35,0.01}{\textbf{\textit{#1}}}}
\newcommand{\ConstantTok}[1]{\textcolor[rgb]{0.00,0.00,0.00}{#1}}
\newcommand{\ControlFlowTok}[1]{\textcolor[rgb]{0.13,0.29,0.53}{\textbf{#1}}}
\newcommand{\DataTypeTok}[1]{\textcolor[rgb]{0.13,0.29,0.53}{#1}}
\newcommand{\DecValTok}[1]{\textcolor[rgb]{0.00,0.00,0.81}{#1}}
\newcommand{\DocumentationTok}[1]{\textcolor[rgb]{0.56,0.35,0.01}{\textbf{\textit{#1}}}}
\newcommand{\ErrorTok}[1]{\textcolor[rgb]{0.64,0.00,0.00}{\textbf{#1}}}
\newcommand{\ExtensionTok}[1]{#1}
\newcommand{\FloatTok}[1]{\textcolor[rgb]{0.00,0.00,0.81}{#1}}
\newcommand{\FunctionTok}[1]{\textcolor[rgb]{0.00,0.00,0.00}{#1}}
\newcommand{\ImportTok}[1]{#1}
\newcommand{\InformationTok}[1]{\textcolor[rgb]{0.56,0.35,0.01}{\textbf{\textit{#1}}}}
\newcommand{\KeywordTok}[1]{\textcolor[rgb]{0.13,0.29,0.53}{\textbf{#1}}}
\newcommand{\NormalTok}[1]{#1}
\newcommand{\OperatorTok}[1]{\textcolor[rgb]{0.81,0.36,0.00}{\textbf{#1}}}
\newcommand{\OtherTok}[1]{\textcolor[rgb]{0.56,0.35,0.01}{#1}}
\newcommand{\PreprocessorTok}[1]{\textcolor[rgb]{0.56,0.35,0.01}{\textit{#1}}}
\newcommand{\RegionMarkerTok}[1]{#1}
\newcommand{\SpecialCharTok}[1]{\textcolor[rgb]{0.00,0.00,0.00}{#1}}
\newcommand{\SpecialStringTok}[1]{\textcolor[rgb]{0.31,0.60,0.02}{#1}}
\newcommand{\StringTok}[1]{\textcolor[rgb]{0.31,0.60,0.02}{#1}}
\newcommand{\VariableTok}[1]{\textcolor[rgb]{0.00,0.00,0.00}{#1}}
\newcommand{\VerbatimStringTok}[1]{\textcolor[rgb]{0.31,0.60,0.02}{#1}}
\newcommand{\WarningTok}[1]{\textcolor[rgb]{0.56,0.35,0.01}{\textbf{\textit{#1}}}}
\usepackage{graphicx,grffile}
\makeatletter
\def\maxwidth{\ifdim\Gin@nat@width>\linewidth\linewidth\else\Gin@nat@width\fi}
\def\maxheight{\ifdim\Gin@nat@height>\textheight\textheight\else\Gin@nat@height\fi}
\makeatother
% Scale images if necessary, so that they will not overflow the page
% margins by default, and it is still possible to overwrite the defaults
% using explicit options in \includegraphics[width, height, ...]{}
\setkeys{Gin}{width=\maxwidth,height=\maxheight,keepaspectratio}
\IfFileExists{parskip.sty}{%
\usepackage{parskip}
}{% else
\setlength{\parindent}{0pt}
\setlength{\parskip}{6pt plus 2pt minus 1pt}
}
\setlength{\emergencystretch}{3em}  % prevent overfull lines
\providecommand{\tightlist}{%
  \setlength{\itemsep}{0pt}\setlength{\parskip}{0pt}}
\setcounter{secnumdepth}{0}
% Redefines (sub)paragraphs to behave more like sections
\ifx\paragraph\undefined\else
\let\oldparagraph\paragraph
\renewcommand{\paragraph}[1]{\oldparagraph{#1}\mbox{}}
\fi
\ifx\subparagraph\undefined\else
\let\oldsubparagraph\subparagraph
\renewcommand{\subparagraph}[1]{\oldsubparagraph{#1}\mbox{}}
\fi

%%% Use protect on footnotes to avoid problems with footnotes in titles
\let\rmarkdownfootnote\footnote%
\def\footnote{\protect\rmarkdownfootnote}

%%% Change title format to be more compact
\usepackage{titling}

% Create subtitle command for use in maketitle
\providecommand{\subtitle}[1]{
  \posttitle{
    \begin{center}\large#1\end{center}
    }
}

\setlength{\droptitle}{-2em}

  \title{MÓDULO 1: HERRAMIENTAS BIG DATA}
    \pretitle{\vspace{\droptitle}\centering\huge}
  \posttitle{\par}
  \subtitle{HERRAMIENTAS DE ANALISIS: PROGRAMACIÓN EN R - MICROACTIVIDADES}
  \author{Nombre Alumno}
    \preauthor{\centering\large\emph}
  \postauthor{\par}
      \predate{\centering\large\emph}
  \postdate{\par}
    \date{Fecha}


\begin{document}
\maketitle

\hypertarget{ejercicio-1}{%
\section{EJERCICIO 1}\label{ejercicio-1}}

\#vamos a jugar

Para el ejercicio 1, utilizaremos los datos los datos \texttt{millas}
que hay el package \texttt{datos}. Estos datos consisten en 238 filas y
11 columnas que describen el consumo de combustible de 38 modelos de
coche populares.

Puedes consultar más sobre los datos en la ayuda: \texttt{?millas}.

\begin{Shaded}
\begin{Highlighting}[]
\KeywordTok{library}\NormalTok{(datos)}
\KeywordTok{suppressPackageStartupMessages}\NormalTok{(}\KeywordTok{library}\NormalTok{(tidyverse))}
\end{Highlighting}
\end{Shaded}

\begin{verbatim}
## Warning: package 'tidyverse' was built under R version 3.6.2
\end{verbatim}

\begin{verbatim}
## Warning: package 'tibble' was built under R version 3.6.2
\end{verbatim}

\begin{verbatim}
## Warning: package 'tidyr' was built under R version 3.6.2
\end{verbatim}

\begin{verbatim}
## Warning: package 'readr' was built under R version 3.6.2
\end{verbatim}

\begin{verbatim}
## Warning: package 'purrr' was built under R version 3.6.2
\end{verbatim}

\begin{verbatim}
## Warning: package 'dplyr' was built under R version 3.6.2
\end{verbatim}

\begin{verbatim}
## Warning: package 'forcats' was built under R version 3.6.2
\end{verbatim}

\begin{verbatim}
?millas
\end{verbatim}

\hypertarget{ejercicio-1.1.}{%
\subsection{EJERCICIO 1.1.}\label{ejercicio-1.1.}}

A partir de los datos de \textbf{millas}, dibuja un gráfico de
dispersión de puntos que muestre las millas recorridas en autopista por
galón de combustible consumido (\textbf{autopista}) respecto a la
\textbf{cilindrada} del motor de cada automóvil. No olvides añadir
títulos al gráfico y a los ejes x e y.

\begin{Shaded}
\begin{Highlighting}[]
\CommentTok{# Solución:}

\NormalTok{ejercicio_}\DecValTok{1}\NormalTok{ <-}\StringTok{ }\KeywordTok{ggplot}\NormalTok{(}\DataTypeTok{data =}\NormalTok{ millas, }\DataTypeTok{mapping =} \KeywordTok{aes}\NormalTok{(}\DataTypeTok{x =}\NormalTok{autopista, }\DataTypeTok{y =}\NormalTok{ cilindrada)) }\OperatorTok{+}\StringTok{ }
\StringTok{  }\KeywordTok{geom_point}\NormalTok{() }\OperatorTok{+}
\StringTok{  }\KeywordTok{labs}\NormalTok{(}\DataTypeTok{title =} \StringTok{"Consumo en función de cilindrada"}
\NormalTok{       , }\DataTypeTok{x =} \StringTok{"autopista"}
\NormalTok{       , }\DataTypeTok{y =} \StringTok{"cilindrada"}\NormalTok{)}
  
\NormalTok{ejercicio_}\DecValTok{1}
\end{Highlighting}
\end{Shaded}

\includegraphics{modulo1_tema4_R_Ejercicios_parteI_files/figure-latex/unnamed-chunk-2-1.pdf}

\hypertarget{ejercicio-1.2.}{%
\subsection{EJERCICIO 1.2.}\label{ejercicio-1.2.}}

A partir del gráfico del ejercicio 1.1., escoge una columna para cada
uno de los siguientes parámetros estéticos: \texttt{color},
\texttt{size} y \texttt{shape}.

\begin{quote}
Truco: Observa que puedes seleccionar tanto columnas numéricas como de
tipo carácter o factor. Si lo crees interesante, puedes utilizar la
misma columna para distintos parámetros del gráfico .
\end{quote}

Comenta algún aspecto relevante que hayas descubierto sobre los coches a
partir del gráfico.

\begin{Shaded}
\begin{Highlighting}[]
\CommentTok{# Solución:}

\NormalTok{ejercicio_}\DecValTok{2}\NormalTok{ <-}\StringTok{ }\KeywordTok{ggplot}\NormalTok{(}\DataTypeTok{data =}\NormalTok{ millas, }\DataTypeTok{mapping =} \KeywordTok{aes}\NormalTok{(}\DataTypeTok{x =}\NormalTok{autopista, }\DataTypeTok{y =}\NormalTok{ cilindrada, }\DataTypeTok{color =}\NormalTok{ modelo, }\DataTypeTok{shape =}\NormalTok{ fabricante, }\DataTypeTok{size =}\NormalTok{ anio)) }\OperatorTok{+}\StringTok{ }
\StringTok{  }\KeywordTok{geom_point}\NormalTok{() }\OperatorTok{+}
\StringTok{  }\KeywordTok{scale_shape_manual}\NormalTok{(}\DataTypeTok{values=}\KeywordTok{seq}\NormalTok{(}\DecValTok{0}\NormalTok{,}\DecValTok{15}\NormalTok{)) }\OperatorTok{+}
\StringTok{  }\KeywordTok{labs}\NormalTok{(}\DataTypeTok{title =} \StringTok{"Consumo en función de cilindrada"}
\NormalTok{       , }\DataTypeTok{x =} \StringTok{"autopista"}
\NormalTok{       , }\DataTypeTok{y =} \StringTok{"cilindrada"}\NormalTok{)}

\NormalTok{ejercicio_}\DecValTok{2}
\end{Highlighting}
\end{Shaded}

\includegraphics{modulo1_tema4_R_Ejercicios_parteI_files/figure-latex/unnamed-chunk-3-1.pdf}

\hypertarget{ejercicio-1.3.}{%
\subsection{EJERCICIO 1.3.}\label{ejercicio-1.3.}}

Transforma el siguiente vector de tipo \texttt{factor} a tipo
\texttt{numeric} de forma que el valor final mostrado sea exactamente el
mismo en ambos vectores, pero con formato distinto. Para ello utiliza
\texttt{as.character()} y \texttt{as.numeric()}.

¿Qué sucede si sólo utilizas \texttt{as.numeric()} directamente sobre la
columna factor?

\begin{Shaded}
\begin{Highlighting}[]
\NormalTok{vec <-}\StringTok{ }\KeywordTok{factor}\NormalTok{(}\KeywordTok{c}\NormalTok{(}\StringTok{"8"}\NormalTok{,}\StringTok{"5"}\NormalTok{,}\StringTok{"9"}\NormalTok{,}\StringTok{"8"}\NormalTok{,}\StringTok{"1"}\NormalTok{,}\StringTok{"7"}\NormalTok{))}
\KeywordTok{print}\NormalTok{(vec) }\CommentTok{# valor mostrado}
\end{Highlighting}
\end{Shaded}

\begin{verbatim}
## [1] 8 5 9 8 1 7
## Levels: 1 5 7 8 9
\end{verbatim}

\begin{Shaded}
\begin{Highlighting}[]
\CommentTok{# Solución:}

\NormalTok{vec_numeric <-}\StringTok{ }\KeywordTok{as.numeric}\NormalTok{(}\KeywordTok{as.character}\NormalTok{(vec))}
\KeywordTok{print}\NormalTok{(vec_numeric)}
\end{Highlighting}
\end{Shaded}

\begin{verbatim}
## [1] 8 5 9 8 1 7
\end{verbatim}

\begin{Shaded}
\begin{Highlighting}[]
\CommentTok{#si solo usamos as.numeric() entonces le asignamos numericamente a cada factor distinto un orden (categorizamos)}
\NormalTok{vec_numeric_}\DecValTok{2}\NormalTok{ <-}\StringTok{ }\KeywordTok{as.numeric}\NormalTok{(vec)}
\KeywordTok{print}\NormalTok{(vec_numeric_}\DecValTok{2}\NormalTok{)}
\end{Highlighting}
\end{Shaded}

\begin{verbatim}
## [1] 4 2 5 4 1 3
\end{verbatim}

\hypertarget{ejercicio-1.4.}{%
\subsection{EJERCICIO 1.4.}\label{ejercicio-1.4.}}

Es millas un objeto de la clase \emph{data.frame} o \emph{matrix}?

¿Y el siguiente objeto \texttt{obj}?

\begin{Shaded}
\begin{Highlighting}[]
\NormalTok{obj1 <-}\StringTok{ }\KeywordTok{cbind}\NormalTok{(millas}\OperatorTok{$}\NormalTok{cilindrada,millas}\OperatorTok{$}\NormalTok{cilindros)}

\CommentTok{# solución}

\KeywordTok{print}\NormalTok{(millas)}
\end{Highlighting}
\end{Shaded}

\begin{verbatim}
## # A tibble: 234 x 11
##    fabricante modelo cilindrada  anio cilindros transmision traccion ciudad
##    <chr>      <chr>       <dbl> <int>     <int> <chr>       <chr>     <int>
##  1 audi       a4            1.8  1999         4 auto(l5)    d            18
##  2 audi       a4            1.8  1999         4 manual(m5)  d            21
##  3 audi       a4            2    2008         4 manual(m6)  d            20
##  4 audi       a4            2    2008         4 auto(av)    d            21
##  5 audi       a4            2.8  1999         6 auto(l5)    d            16
##  6 audi       a4            2.8  1999         6 manual(m5)  d            18
##  7 audi       a4            3.1  2008         6 auto(av)    d            18
##  8 audi       a4 qu~        1.8  1999         4 manual(m5)  4            18
##  9 audi       a4 qu~        1.8  1999         4 auto(l5)    4            16
## 10 audi       a4 qu~        2    2008         4 manual(m6)  4            20
## # ... with 224 more rows, and 3 more variables: autopista <int>,
## #   combustible <chr>, clase <chr>
\end{verbatim}

\begin{Shaded}
\begin{Highlighting}[]
\KeywordTok{class}\NormalTok{(millas)}
\end{Highlighting}
\end{Shaded}

\begin{verbatim}
## [1] "tbl_df"     "tbl"        "data.frame"
\end{verbatim}

\begin{Shaded}
\begin{Highlighting}[]
\KeywordTok{print}\NormalTok{(obj1)}
\end{Highlighting}
\end{Shaded}

\begin{verbatim}
##        [,1] [,2]
##   [1,]  1.8    4
##   [2,]  1.8    4
##   [3,]  2.0    4
##   [4,]  2.0    4
##   [5,]  2.8    6
##   [6,]  2.8    6
##   [7,]  3.1    6
##   [8,]  1.8    4
##   [9,]  1.8    4
##  [10,]  2.0    4
##  [11,]  2.0    4
##  [12,]  2.8    6
##  [13,]  2.8    6
##  [14,]  3.1    6
##  [15,]  3.1    6
##  [16,]  2.8    6
##  [17,]  3.1    6
##  [18,]  4.2    8
##  [19,]  5.3    8
##  [20,]  5.3    8
##  [21,]  5.3    8
##  [22,]  5.7    8
##  [23,]  6.0    8
##  [24,]  5.7    8
##  [25,]  5.7    8
##  [26,]  6.2    8
##  [27,]  6.2    8
##  [28,]  7.0    8
##  [29,]  5.3    8
##  [30,]  5.3    8
##  [31,]  5.7    8
##  [32,]  6.5    8
##  [33,]  2.4    4
##  [34,]  2.4    4
##  [35,]  3.1    6
##  [36,]  3.5    6
##  [37,]  3.6    6
##  [38,]  2.4    4
##  [39,]  3.0    6
##  [40,]  3.3    6
##  [41,]  3.3    6
##  [42,]  3.3    6
##  [43,]  3.3    6
##  [44,]  3.3    6
##  [45,]  3.8    6
##  [46,]  3.8    6
##  [47,]  3.8    6
##  [48,]  4.0    6
##  [49,]  3.7    6
##  [50,]  3.7    6
##  [51,]  3.9    6
##  [52,]  3.9    6
##  [53,]  4.7    8
##  [54,]  4.7    8
##  [55,]  4.7    8
##  [56,]  5.2    8
##  [57,]  5.2    8
##  [58,]  3.9    6
##  [59,]  4.7    8
##  [60,]  4.7    8
##  [61,]  4.7    8
##  [62,]  5.2    8
##  [63,]  5.7    8
##  [64,]  5.9    8
##  [65,]  4.7    8
##  [66,]  4.7    8
##  [67,]  4.7    8
##  [68,]  4.7    8
##  [69,]  4.7    8
##  [70,]  4.7    8
##  [71,]  5.2    8
##  [72,]  5.2    8
##  [73,]  5.7    8
##  [74,]  5.9    8
##  [75,]  4.6    8
##  [76,]  5.4    8
##  [77,]  5.4    8
##  [78,]  4.0    6
##  [79,]  4.0    6
##  [80,]  4.0    6
##  [81,]  4.0    6
##  [82,]  4.6    8
##  [83,]  5.0    8
##  [84,]  4.2    6
##  [85,]  4.2    6
##  [86,]  4.6    8
##  [87,]  4.6    8
##  [88,]  4.6    8
##  [89,]  5.4    8
##  [90,]  5.4    8
##  [91,]  3.8    6
##  [92,]  3.8    6
##  [93,]  4.0    6
##  [94,]  4.0    6
##  [95,]  4.6    8
##  [96,]  4.6    8
##  [97,]  4.6    8
##  [98,]  4.6    8
##  [99,]  5.4    8
## [100,]  1.6    4
## [101,]  1.6    4
## [102,]  1.6    4
## [103,]  1.6    4
## [104,]  1.6    4
## [105,]  1.8    4
## [106,]  1.8    4
## [107,]  1.8    4
## [108,]  2.0    4
## [109,]  2.4    4
## [110,]  2.4    4
## [111,]  2.4    4
## [112,]  2.4    4
## [113,]  2.5    6
## [114,]  2.5    6
## [115,]  3.3    6
## [116,]  2.0    4
## [117,]  2.0    4
## [118,]  2.0    4
## [119,]  2.0    4
## [120,]  2.7    6
## [121,]  2.7    6
## [122,]  2.7    6
## [123,]  3.0    6
## [124,]  3.7    6
## [125,]  4.0    6
## [126,]  4.7    8
## [127,]  4.7    8
## [128,]  4.7    8
## [129,]  5.7    8
## [130,]  6.1    8
## [131,]  4.0    8
## [132,]  4.2    8
## [133,]  4.4    8
## [134,]  4.6    8
## [135,]  5.4    8
## [136,]  5.4    8
## [137,]  5.4    8
## [138,]  4.0    6
## [139,]  4.0    6
## [140,]  4.6    8
## [141,]  5.0    8
## [142,]  2.4    4
## [143,]  2.4    4
## [144,]  2.5    4
## [145,]  2.5    4
## [146,]  3.5    6
## [147,]  3.5    6
## [148,]  3.0    6
## [149,]  3.0    6
## [150,]  3.5    6
## [151,]  3.3    6
## [152,]  3.3    6
## [153,]  4.0    6
## [154,]  5.6    8
## [155,]  3.1    6
## [156,]  3.8    6
## [157,]  3.8    6
## [158,]  3.8    6
## [159,]  5.3    8
## [160,]  2.5    4
## [161,]  2.5    4
## [162,]  2.5    4
## [163,]  2.5    4
## [164,]  2.5    4
## [165,]  2.5    4
## [166,]  2.2    4
## [167,]  2.2    4
## [168,]  2.5    4
## [169,]  2.5    4
## [170,]  2.5    4
## [171,]  2.5    4
## [172,]  2.5    4
## [173,]  2.5    4
## [174,]  2.7    4
## [175,]  2.7    4
## [176,]  3.4    6
## [177,]  3.4    6
## [178,]  4.0    6
## [179,]  4.7    8
## [180,]  2.2    4
## [181,]  2.2    4
## [182,]  2.4    4
## [183,]  2.4    4
## [184,]  3.0    6
## [185,]  3.0    6
## [186,]  3.5    6
## [187,]  2.2    4
## [188,]  2.2    4
## [189,]  2.4    4
## [190,]  2.4    4
## [191,]  3.0    6
## [192,]  3.0    6
## [193,]  3.3    6
## [194,]  1.8    4
## [195,]  1.8    4
## [196,]  1.8    4
## [197,]  1.8    4
## [198,]  1.8    4
## [199,]  4.7    8
## [200,]  5.7    8
## [201,]  2.7    4
## [202,]  2.7    4
## [203,]  2.7    4
## [204,]  3.4    6
## [205,]  3.4    6
## [206,]  4.0    6
## [207,]  4.0    6
## [208,]  2.0    4
## [209,]  2.0    4
## [210,]  2.0    4
## [211,]  2.0    4
## [212,]  2.8    6
## [213,]  1.9    4
## [214,]  2.0    4
## [215,]  2.0    4
## [216,]  2.0    4
## [217,]  2.0    4
## [218,]  2.5    5
## [219,]  2.5    5
## [220,]  2.8    6
## [221,]  2.8    6
## [222,]  1.9    4
## [223,]  1.9    4
## [224,]  2.0    4
## [225,]  2.0    4
## [226,]  2.5    5
## [227,]  2.5    5
## [228,]  1.8    4
## [229,]  1.8    4
## [230,]  2.0    4
## [231,]  2.0    4
## [232,]  2.8    6
## [233,]  2.8    6
## [234,]  3.6    6
\end{verbatim}

\begin{Shaded}
\begin{Highlighting}[]
\KeywordTok{class}\NormalTok{(obj1)}
\end{Highlighting}
\end{Shaded}

\begin{verbatim}
## [1] "matrix"
\end{verbatim}

\hypertarget{ejercicio-1.5.}{%
\subsection{EJERCICIO 1.5.}\label{ejercicio-1.5.}}

Crea una función que tome un vector de tipo integer como input y retorne
un objetido de tipo lista que contega los siguientes elementos:

\begin{enumerate}
\def\labelenumi{\arabic{enumi}.}
\tightlist
\item
  El último valor del vector
\item
  Los elementos de las posiciones impares.
\item
  Todos los elementos excepto el primero.
\item
  Solo números impares (y no valores faltantes).
\end{enumerate}

\begin{Shaded}
\begin{Highlighting}[]
\CommentTok{# solución}

\NormalTok{ejercicio_}\DecValTok{5}\NormalTok{ <-}\StringTok{ }\ControlFlowTok{function}\NormalTok{(input) \{}
  \ControlFlowTok{if}\NormalTok{ (input }\OperatorTok{==}\StringTok{ }\DecValTok{1}\NormalTok{) \{}
\NormalTok{    function_return <-}\StringTok{ }\NormalTok{vec[}\KeywordTok{length}\NormalTok{(vec)]}
\NormalTok{  \}}
  \ControlFlowTok{if}\NormalTok{ (input }\OperatorTok{==}\StringTok{ }\DecValTok{2}\NormalTok{)\{}
\NormalTok{    function_return <-}\StringTok{ }\NormalTok{vec[}\KeywordTok{c}\NormalTok{(}\DecValTok{1}\NormalTok{,}\DecValTok{3}\NormalTok{,}\DecValTok{5}\NormalTok{)]}
\NormalTok{  \}}
  \ControlFlowTok{if}\NormalTok{ (input }\OperatorTok{==}\StringTok{ }\DecValTok{3}\NormalTok{)\{}
\NormalTok{    function_return <-}\StringTok{ }\NormalTok{vec[}\DecValTok{2}\OperatorTok{:}\KeywordTok{length}\NormalTok{(vec)]}
\NormalTok{  \}}
  
  \ControlFlowTok{if}\NormalTok{ (input }\OperatorTok{==}\StringTok{ }\DecValTok{4}\NormalTok{)\{}
\NormalTok{    numeric_vector <-}\StringTok{ }\KeywordTok{as.numeric}\NormalTok{(}\KeywordTok{as.character}\NormalTok{(vec))}
\NormalTok{    function_return <-}\StringTok{ }\KeywordTok{subset}\NormalTok{(numeric_vector, numeric_vector }\OperatorTok\StringTok{ }\DecValTok{2} \OperatorTok{!=}\StringTok{ }\DecValTok{0}\NormalTok{)}
\NormalTok{  \}}
   \KeywordTok{return}\NormalTok{(function_return)}
    

\NormalTok{\}}

\KeywordTok{ejercicio_5}\NormalTok{(}\DecValTok{1}\NormalTok{)}
\end{Highlighting}
\end{Shaded}

\begin{verbatim}
## [1] 7
## Levels: 1 5 7 8 9
\end{verbatim}

\begin{Shaded}
\begin{Highlighting}[]
\KeywordTok{ejercicio_5}\NormalTok{(}\DecValTok{2}\NormalTok{)}
\end{Highlighting}
\end{Shaded}

\begin{verbatim}
## [1] 8 9 1
## Levels: 1 5 7 8 9
\end{verbatim}

\begin{Shaded}
\begin{Highlighting}[]
\KeywordTok{ejercicio_5}\NormalTok{(}\DecValTok{3}\NormalTok{)}
\end{Highlighting}
\end{Shaded}

\begin{verbatim}
## [1] 5 9 8 1 7
## Levels: 1 5 7 8 9
\end{verbatim}

\begin{Shaded}
\begin{Highlighting}[]
\KeywordTok{ejercicio_5}\NormalTok{(}\DecValTok{4}\NormalTok{)}
\end{Highlighting}
\end{Shaded}

\begin{verbatim}
## [1] 5 9 1 7
\end{verbatim}

\hypertarget{ejercicio-1.6.}{%
\subsection{EJERCICIO 1.6.}\label{ejercicio-1.6.}}

Busca un ejemplo de objeto x en el que la expresión
\texttt{x{[}-which(x\ \textgreater{}\ 0){]}} no devuelve el mismo
resultado que \texttt{x{[}x\ \textless{}=\ 0{]}}

\begin{Shaded}
\begin{Highlighting}[]
\CommentTok{# Solución:}

\NormalTok{x <-}\StringTok{ }\KeywordTok{c}\NormalTok{(}\OtherTok{NaN}\NormalTok{, }\DecValTok{2}\NormalTok{)}

\NormalTok{x[}\OperatorTok{-}\KeywordTok{which}\NormalTok{(x }\OperatorTok{>}\StringTok{ }\DecValTok{0}\NormalTok{)]}
\end{Highlighting}
\end{Shaded}

\begin{verbatim}
## [1] NaN
\end{verbatim}

\begin{Shaded}
\begin{Highlighting}[]
\NormalTok{x[x }\OperatorTok{<=}\StringTok{ }\DecValTok{0}\NormalTok{]}
\end{Highlighting}
\end{Shaded}

\begin{verbatim}
## [1] NA
\end{verbatim}

\hypertarget{ejercicio-1.7.}{%
\subsection{EJERCICIO 1.7.}\label{ejercicio-1.7.}}

Añade a millas una nueva columna llamada ``fabr\_mod'' que contenga la
concatenación del nombre del fabricante, un guion ``-'' y el modelo del
coche. Presenta la nueva columna mediante la función head().

\begin{Shaded}
\begin{Highlighting}[]
\CommentTok{# Solución:}

\NormalTok{millas}\OperatorTok{$}\NormalTok{fabr_mod <-}\StringTok{ }\DecValTok{1}\OperatorTok{:}\KeywordTok{nrow}\NormalTok{(millas)}


\NormalTok{millas}\OperatorTok{$}\NormalTok{fabr_mod <-}\StringTok{ }\KeywordTok{paste}\NormalTok{(millas}\OperatorTok{$}\NormalTok{fabricante,}\StringTok{"-"}\NormalTok{,millas}\OperatorTok{$}\NormalTok{modelo)}
\KeywordTok{head}\NormalTok{(millas)}
\end{Highlighting}
\end{Shaded}

\begin{verbatim}
## # A tibble: 6 x 12
##   fabricante modelo cilindrada  anio cilindros transmision traccion ciudad
##   <chr>      <chr>       <dbl> <int>     <int> <chr>       <chr>     <int>
## 1 audi       a4            1.8  1999         4 auto(l5)    d            18
## 2 audi       a4            1.8  1999         4 manual(m5)  d            21
## 3 audi       a4            2    2008         4 manual(m6)  d            20
## 4 audi       a4            2    2008         4 auto(av)    d            21
## 5 audi       a4            2.8  1999         6 auto(l5)    d            16
## 6 audi       a4            2.8  1999         6 manual(m5)  d            18
## # ... with 4 more variables: autopista <int>, combustible <chr>,
## #   clase <chr>, fabr_mod <chr>
\end{verbatim}

\hypertarget{ejercicio-1.8.}{%
\subsection{EJERCICIO 1.8.}\label{ejercicio-1.8.}}

Selecciona todos los coches de \texttt{millas} que cumplan con todas
todas las condiciones siguientes:

\begin{itemize}
\tightlist
\item
  La marca es distinta a ``dodge''
\item
  Tiene tracción en las cuatro puertas
\item
  Han estado fabricados antes del 2008
\item
  Las millas/galón, o bién en ciudad, o bién en carretera, no llegan a
  12 millas/galón.
\end{itemize}

¿Cuantos coches has encontrado?

\begin{Shaded}
\begin{Highlighting}[]
\CommentTok{# Solución:}

\NormalTok{ejercicio_}\DecValTok{8}\NormalTok{ <-}\StringTok{ }\KeywordTok{subset}\NormalTok{(millas,fabricante }\OperatorTok{!=}\StringTok{ "dodge"} \OperatorTok{&}\StringTok{ }\NormalTok{traccion }\OperatorTok{==}\StringTok{ "4"} \OperatorTok{&}\StringTok{ }\NormalTok{anio }\OperatorTok{<}\StringTok{ }\DecValTok{2008} \OperatorTok{&}\StringTok{ }\NormalTok{(ciudad }\OperatorTok{<=}\StringTok{ }\DecValTok{12} \OperatorTok{|}\StringTok{ }\NormalTok{autopista }\OperatorTok{<=}\StringTok{ }\DecValTok{12}\NormalTok{))}
\NormalTok{ejercicio_}\DecValTok{8}
\end{Highlighting}
\end{Shaded}

\begin{verbatim}
## # A tibble: 5 x 12
##   fabricante modelo  cilindrada  anio cilindros transmision traccion ciudad
##   <chr>      <chr>        <dbl> <int>     <int> <chr>       <chr>     <int>
## 1 chevrolet  k1500 ~        5.7  1999         8 auto(l4)    4            11
## 2 ford       f150 p~        5.4  1999         8 auto(l4)    4            11
## 3 land rover range ~        4    1999         8 auto(l4)    4            11
## 4 land rover range ~        4.6  1999         8 auto(l4)    4            11
## 5 toyota     land c~        4.7  1999         8 auto(l4)    4            11
## # ... with 4 more variables: autopista <int>, combustible <chr>,
## #   clase <chr>, fabr_mod <chr>
\end{verbatim}

\hypertarget{ejercicio-1.9.}{%
\subsection{EJERCICIO 1.9.}\label{ejercicio-1.9.}}

Añade una nueva columna ``vol\_por\_cil'' a \texttt{obj} del ejercicio
1.4. que contenga el ratio de la cilindrada sobre el número de
cilindros. Presenta el summary de la nueva columna.

\begin{Shaded}
\begin{Highlighting}[]
\CommentTok{# Solución:}

\NormalTok{obj <-}\StringTok{ }\KeywordTok{as.data.frame}\NormalTok{(}\KeywordTok{cbind}\NormalTok{(millas}\OperatorTok{$}\NormalTok{cilindrada,millas}\OperatorTok{$}\NormalTok{cilindros, }\OtherTok{NA}\NormalTok{))}
\KeywordTok{colnames}\NormalTok{(obj) <-}\StringTok{ }\KeywordTok{c}\NormalTok{(}\StringTok{"cilindrada"}\NormalTok{,}\StringTok{"cilindros"}\NormalTok{,}\StringTok{"vol_por_cil"}\NormalTok{)}
\NormalTok{obj}\OperatorTok{$}\NormalTok{vol_por_cil <-}\StringTok{ }\KeywordTok{as.numeric}\NormalTok{(obj}\OperatorTok{$}\NormalTok{cilindrada) }\OperatorTok{/}\StringTok{ }\KeywordTok{as.numeric}\NormalTok{(obj}\OperatorTok{$}\NormalTok{cilindros)}
\KeywordTok{summary}\NormalTok{(obj}\OperatorTok{$}\NormalTok{vol_por_cil)}
\end{Highlighting}
\end{Shaded}

\begin{verbatim}
##    Min. 1st Qu.  Median    Mean 3rd Qu.    Max. 
##  0.4000  0.5000  0.5875  0.5780  0.6500  0.8750
\end{verbatim}

\hypertarget{ejercicio-1.10.}{%
\subsection{EJERCICIO 1.10.}\label{ejercicio-1.10.}}

Modifica los valores de la columna ``vol\_por\_cil'' del objeto
\texttt{obj} del ejercicio 1.9. asignando NA a los valores de esta
columna que sean superiores a 0.7.

Presenta los datos con un summary del nuevo objeto \texttt{obj}.
¿Cuántos valores NA se han creado en esta columna?

\begin{Shaded}
\begin{Highlighting}[]
\CommentTok{# Solución:}

\NormalTok{ejercicio_}\DecValTok{10}\NormalTok{ <-}\StringTok{ }\NormalTok{obj}

\NormalTok{ejercicio_}\DecValTok{10}\OperatorTok{$}\NormalTok{vol_por_cil <-}\StringTok{ }\KeywordTok{ifelse}\NormalTok{ ( ejercicio_}\DecValTok{10}\OperatorTok{$}\NormalTok{vol_por_cil }\OperatorTok{>=}\StringTok{ }\FloatTok{0.7}\NormalTok{, }\OtherTok{NA}\NormalTok{, ejercicio_}\DecValTok{10}\OperatorTok{$}\NormalTok{vol_por_cil)}
\KeywordTok{print}\NormalTok{(ejercicio_}\DecValTok{10}\NormalTok{)}
\end{Highlighting}
\end{Shaded}

\begin{verbatim}
##     cilindrada cilindros vol_por_cil
## 1          1.8         4   0.4500000
## 2          1.8         4   0.4500000
## 3          2.0         4   0.5000000
## 4          2.0         4   0.5000000
## 5          2.8         6   0.4666667
## 6          2.8         6   0.4666667
## 7          3.1         6   0.5166667
## 8          1.8         4   0.4500000
## 9          1.8         4   0.4500000
## 10         2.0         4   0.5000000
## 11         2.0         4   0.5000000
## 12         2.8         6   0.4666667
## 13         2.8         6   0.4666667
## 14         3.1         6   0.5166667
## 15         3.1         6   0.5166667
## 16         2.8         6   0.4666667
## 17         3.1         6   0.5166667
## 18         4.2         8   0.5250000
## 19         5.3         8   0.6625000
## 20         5.3         8   0.6625000
## 21         5.3         8   0.6625000
## 22         5.7         8          NA
## 23         6.0         8          NA
## 24         5.7         8          NA
## 25         5.7         8          NA
## 26         6.2         8          NA
## 27         6.2         8          NA
## 28         7.0         8          NA
## 29         5.3         8   0.6625000
## 30         5.3         8   0.6625000
## 31         5.7         8          NA
## 32         6.5         8          NA
## 33         2.4         4   0.6000000
## 34         2.4         4   0.6000000
## 35         3.1         6   0.5166667
## 36         3.5         6   0.5833333
## 37         3.6         6   0.6000000
## 38         2.4         4   0.6000000
## 39         3.0         6   0.5000000
## 40         3.3         6   0.5500000
## 41         3.3         6   0.5500000
## 42         3.3         6   0.5500000
## 43         3.3         6   0.5500000
## 44         3.3         6   0.5500000
## 45         3.8         6   0.6333333
## 46         3.8         6   0.6333333
## 47         3.8         6   0.6333333
## 48         4.0         6   0.6666667
## 49         3.7         6   0.6166667
## 50         3.7         6   0.6166667
## 51         3.9         6   0.6500000
## 52         3.9         6   0.6500000
## 53         4.7         8   0.5875000
## 54         4.7         8   0.5875000
## 55         4.7         8   0.5875000
## 56         5.2         8   0.6500000
## 57         5.2         8   0.6500000
## 58         3.9         6   0.6500000
## 59         4.7         8   0.5875000
## 60         4.7         8   0.5875000
## 61         4.7         8   0.5875000
## 62         5.2         8   0.6500000
## 63         5.7         8          NA
## 64         5.9         8          NA
## 65         4.7         8   0.5875000
## 66         4.7         8   0.5875000
## 67         4.7         8   0.5875000
## 68         4.7         8   0.5875000
## 69         4.7         8   0.5875000
## 70         4.7         8   0.5875000
## 71         5.2         8   0.6500000
## 72         5.2         8   0.6500000
## 73         5.7         8          NA
## 74         5.9         8          NA
## 75         4.6         8   0.5750000
## 76         5.4         8   0.6750000
## 77         5.4         8   0.6750000
## 78         4.0         6   0.6666667
## 79         4.0         6   0.6666667
## 80         4.0         6   0.6666667
## 81         4.0         6   0.6666667
## 82         4.6         8   0.5750000
## 83         5.0         8   0.6250000
## 84         4.2         6          NA
## 85         4.2         6          NA
## 86         4.6         8   0.5750000
## 87         4.6         8   0.5750000
## 88         4.6         8   0.5750000
## 89         5.4         8   0.6750000
## 90         5.4         8   0.6750000
## 91         3.8         6   0.6333333
## 92         3.8         6   0.6333333
## 93         4.0         6   0.6666667
## 94         4.0         6   0.6666667
## 95         4.6         8   0.5750000
## 96         4.6         8   0.5750000
## 97         4.6         8   0.5750000
## 98         4.6         8   0.5750000
## 99         5.4         8   0.6750000
## 100        1.6         4   0.4000000
## 101        1.6         4   0.4000000
## 102        1.6         4   0.4000000
## 103        1.6         4   0.4000000
## 104        1.6         4   0.4000000
## 105        1.8         4   0.4500000
## 106        1.8         4   0.4500000
## 107        1.8         4   0.4500000
## 108        2.0         4   0.5000000
## 109        2.4         4   0.6000000
## 110        2.4         4   0.6000000
## 111        2.4         4   0.6000000
## 112        2.4         4   0.6000000
## 113        2.5         6   0.4166667
## 114        2.5         6   0.4166667
## 115        3.3         6   0.5500000
## 116        2.0         4   0.5000000
## 117        2.0         4   0.5000000
## 118        2.0         4   0.5000000
## 119        2.0         4   0.5000000
## 120        2.7         6   0.4500000
## 121        2.7         6   0.4500000
## 122        2.7         6   0.4500000
## 123        3.0         6   0.5000000
## 124        3.7         6   0.6166667
## 125        4.0         6   0.6666667
## 126        4.7         8   0.5875000
## 127        4.7         8   0.5875000
## 128        4.7         8   0.5875000
## 129        5.7         8          NA
## 130        6.1         8          NA
## 131        4.0         8   0.5000000
## 132        4.2         8   0.5250000
## 133        4.4         8   0.5500000
## 134        4.6         8   0.5750000
## 135        5.4         8   0.6750000
## 136        5.4         8   0.6750000
## 137        5.4         8   0.6750000
## 138        4.0         6   0.6666667
## 139        4.0         6   0.6666667
## 140        4.6         8   0.5750000
## 141        5.0         8   0.6250000
## 142        2.4         4   0.6000000
## 143        2.4         4   0.6000000
## 144        2.5         4   0.6250000
## 145        2.5         4   0.6250000
## 146        3.5         6   0.5833333
## 147        3.5         6   0.5833333
## 148        3.0         6   0.5000000
## 149        3.0         6   0.5000000
## 150        3.5         6   0.5833333
## 151        3.3         6   0.5500000
## 152        3.3         6   0.5500000
## 153        4.0         6   0.6666667
## 154        5.6         8          NA
## 155        3.1         6   0.5166667
## 156        3.8         6   0.6333333
## 157        3.8         6   0.6333333
## 158        3.8         6   0.6333333
## 159        5.3         8   0.6625000
## 160        2.5         4   0.6250000
## 161        2.5         4   0.6250000
## 162        2.5         4   0.6250000
## 163        2.5         4   0.6250000
## 164        2.5         4   0.6250000
## 165        2.5         4   0.6250000
## 166        2.2         4   0.5500000
## 167        2.2         4   0.5500000
## 168        2.5         4   0.6250000
## 169        2.5         4   0.6250000
## 170        2.5         4   0.6250000
## 171        2.5         4   0.6250000
## 172        2.5         4   0.6250000
## 173        2.5         4   0.6250000
## 174        2.7         4   0.6750000
## 175        2.7         4   0.6750000
## 176        3.4         6   0.5666667
## 177        3.4         6   0.5666667
## 178        4.0         6   0.6666667
## 179        4.7         8   0.5875000
## 180        2.2         4   0.5500000
## 181        2.2         4   0.5500000
## 182        2.4         4   0.6000000
## 183        2.4         4   0.6000000
## 184        3.0         6   0.5000000
## 185        3.0         6   0.5000000
## 186        3.5         6   0.5833333
## 187        2.2         4   0.5500000
## 188        2.2         4   0.5500000
## 189        2.4         4   0.6000000
## 190        2.4         4   0.6000000
## 191        3.0         6   0.5000000
## 192        3.0         6   0.5000000
## 193        3.3         6   0.5500000
## 194        1.8         4   0.4500000
## 195        1.8         4   0.4500000
## 196        1.8         4   0.4500000
## 197        1.8         4   0.4500000
## 198        1.8         4   0.4500000
## 199        4.7         8   0.5875000
## 200        5.7         8          NA
## 201        2.7         4   0.6750000
## 202        2.7         4   0.6750000
## 203        2.7         4   0.6750000
## 204        3.4         6   0.5666667
## 205        3.4         6   0.5666667
## 206        4.0         6   0.6666667
## 207        4.0         6   0.6666667
## 208        2.0         4   0.5000000
## 209        2.0         4   0.5000000
## 210        2.0         4   0.5000000
## 211        2.0         4   0.5000000
## 212        2.8         6   0.4666667
## 213        1.9         4   0.4750000
## 214        2.0         4   0.5000000
## 215        2.0         4   0.5000000
## 216        2.0         4   0.5000000
## 217        2.0         4   0.5000000
## 218        2.5         5   0.5000000
## 219        2.5         5   0.5000000
## 220        2.8         6   0.4666667
## 221        2.8         6   0.4666667
## 222        1.9         4   0.4750000
## 223        1.9         4   0.4750000
## 224        2.0         4   0.5000000
## 225        2.0         4   0.5000000
## 226        2.5         5   0.5000000
## 227        2.5         5   0.5000000
## 228        1.8         4   0.4500000
## 229        1.8         4   0.4500000
## 230        2.0         4   0.5000000
## 231        2.0         4   0.5000000
## 232        2.8         6   0.4666667
## 233        2.8         6   0.4666667
## 234        3.6         6   0.6000000
\end{verbatim}

\begin{Shaded}
\begin{Highlighting}[]
\NormalTok{ejercio_}\DecValTok{10}\NormalTok{_}\DecValTok{1}\NormalTok{ <-}\StringTok{ }\KeywordTok{subset}\NormalTok{(ejercicio_}\DecValTok{10}\NormalTok{, }\KeywordTok{is.na}\NormalTok{(ejercicio_}\DecValTok{10}\OperatorTok{$}\NormalTok{vol_por_cil))}
\KeywordTok{print}\NormalTok{(}\KeywordTok{nrow}\NormalTok{(ejercio_}\DecValTok{10}\NormalTok{_}\DecValTok{1}\NormalTok{))}
\end{Highlighting}
\end{Shaded}

\begin{verbatim}
## [1] 19
\end{verbatim}

\begin{Shaded}
\begin{Highlighting}[]
\KeywordTok{summary}\NormalTok{(ejercicio_}\DecValTok{10}\OperatorTok{$}\NormalTok{vol_por_cil)}
\end{Highlighting}
\end{Shaded}

\begin{verbatim}
##    Min. 1st Qu.  Median    Mean 3rd Qu.    Max.    NA's 
##  0.4000  0.5000  0.5750  0.5638  0.6250  0.6750      19
\end{verbatim}


\end{document}
